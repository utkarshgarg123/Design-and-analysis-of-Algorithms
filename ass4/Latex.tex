\documentclass[conference]{IEEEtran}
\documentclass{article}
\IEEEoverridecommandlockouts
% The preceding line is only needed to identify funding in the first footnote. If that is unneeded, please comment it out.
\usepackage{cite}
\usepackage{amsmath,amssymb,amsfonts}
\usepackage{graphicx}
\usepackage{textcomp}
\usepackage{xcolor}
\usepackage{titlesec}
\usepackage{textcomp}
\usepackage{epsfig}
\usepackage{algpseudocode}
\usepackage{pgfplots}
\usepackage{tikz}
\usepackage{hyperref}
\usepackage{graphicx}
\pgfplotsset{width=10cm,compat=1.9}
 \usepgfplotslibrary{external}
 \graphicspath{ {./images/} }



\usepackage[linesnumbered,ruled,vlined]{algorithm2e}
\def\BibTeX{{\rm B\kern-.05em{\sc i\kern-.025em b}\kern-.08em
    T\kern-.1667em\lower.7ex\hbox{E}\kern-.125emX}}

\usepackage[ruled,vlined]{algorithm2e}

\tikzexternalize 
\begin{document}

\title{Specific Sequences of given length  \\
\text{\Large{DAA ASSIGNMENT-4 , GROUP 8}}
}
\author{\IEEEauthorblockN{Swaraj Bhosle}
\IEEEauthorblockA{ \text{IIT2019024}}
\and
\IEEEauthorblockN{Ritesh Raj}
\IEEEauthorblockA{ \text{IIT2019025}}
\and
\IEEEauthorblockN{Utkarsh Garg}
\IEEEauthorblockA{ \text{IIT2019026}}
}

\maketitle
{\textbf{\textit{Abstract: In this Paper we have devised an algorithm to find the number of possible sequences of specific length (let say n) such that each of the next element is greater than or equal to twice of the previous element but less than or equal to specific number (let say m).\\ 
}}}
\maketitle

\section{INTRODUCTION}\\
In this report, we are going to discuss about the particular sequence of given length. We are given two integers m and n.
We have to find the number of possible sequences of length n such that each of the next element is greater than or equal to twice of the previous element but less than or equal to m.\\
There should be n elements and value of last element should be at-most m. 
we are dividing the problem into sub problems and again dividing that sub problem in sub problems.basically we are using Divide and Conquer method.\\
\\ This report further contains - 
\renewcommand{\labelenumi}{\Roman{enumi}}
\begin{enumerate}
\setcounter{enumi}{1}
    \item Algorithm Design.
    \item Algorithm And  Analysis.
    \item Time Calculation
    \item Time Complexity
    \item References\\
\end{enumerate}

\section{ALGORITHM DESIGN}\\
We are given two integers as input i.e. \begin{math}
M\end{math} and \begin{math}
N\end{math}. where \begin{math}
N\end{math} is the length of sequence and \begin{math}
M\end{math} is the number where greatest number in the sequence is less than or equal to \begin{math}
M\end{math}. if \begin{math}
M\end{math} is present in the sequence then it must be present at the last element of sequence. As per the given condition the n-th value of the sequence can be at most \begin{math}
M\end{math}.\\\\
Now there are two possible cases for the n-th element:
\renewcommand{\labelenumi}{\alph{enumi}}
\begin{enumerate}
 \setcounter{enumi}{0}
    \item If the last element i.e. n-th element is m, then the \begin{math}
(N-1)th\end{math} element is at most \begin{math}
M/2\end{math}. So We recur for \begin{math}
M/2\end{math} and \begin{math}
(N-1)\end{math}.
    \item If the last element is not m, then the last element is at most m-1. We recur for \begin{math}
(M-1)\end{math} and \begin{math}
N\end{math}.\\
\end{enumerate}
The total number of sequences is the sum of the number of sequences including m and the number of sequences where m is not included.\\ 
Thus the original problem of finding number of sequences of length N with max value M can be subdivided into independent sub problems of finding number of sequences of length N with max value M-1 and number of sequences of length N-1 with max value M/2.\\
\\\textbf{Step1 :}
take \begin{math}
2\end{math} numbers \begin{math}
M\end{math} and \begin{math}
N\end{math} as input from user .\begin{math}
N\end{math} is length of sequence and no other element in the sequence is greater than \begin{math}
M\end{math}. if \begin{math}
M\end{math} is present in the sequence then it is present only at the last position. 
\\\\\textbf{Step2 :}
call the function:\\ 
TotalNumberOfSequence(\begin{math}
M\end{math}, \begin{math}
N\end{math})
\\\\\textbf{Step3 :}
in the function, check whether \begin{math}M\end{math} is greater than \begin{math}N\end{math} or not. If m is smaller than \begin{math}N\end{math} then return 0, as  n is more than the maximum value \begin{math}
M\end{math}.\\
If \begin{math}N\end{math} is 0, then the sequence is empty,  Return 1
\\\\\textbf{Step4 :}
now there are 2 possibilities :
\begin{enumerate}
    \item Reduce  last element value i.e. call the function TotalNumberOfSequence(\begin{math}
M-1\end{math}, n)
    \item Consider last element as \begin{math}M\end{math} and reduce number of terms i.e. call the function TotalNumberOfSequence(\begin{math}
M/2\end{math}, \begin{math}N-1\end{math})
\end{enumerate}
\\\\\textbf{Step5 :}
print the total no of sequences 

\graphicspath{ {./images/} }
\\\\Recursive Tree for \begin{math}M=10\end{math} and \begin{math}N\end{math} = 4
\includegraphics[scale=0.35]{One}



\section{ALGORITHM AND ANALYSIS}\\

\begin{algorithm}[H] 
    \caption{Algorithm 1: }
    \KwIn{two numbers \begin{math}
M\end{math} and \begin{math}
N\end{math}}
    \KwOut {Total number of possible sequences}
    \DontPrintSemicolon
    \SetKwFunction{FMain}{TotalSequences}
    \SetKwProg{Fn}{Function}{:}{}
    \Fn{\FMain{$int$ $m$, $int$ $n$}}{
   
  \If{$M < N$}{
            \KwRet $0$}
        \If{$N == 1$}{
            \KwRet $1$}
        \If{$dp[M][N] != -1$}{
            \KwRet $DP[M][N]$}
   \\\KwRet TotalSequences (\begin{math}M-1\end{math},\begin{math}N\end{math}) + TotalSequences (\begin{math}M/2\end{math}, \begin{math}N-1\end{math})\;
   
   }
    \SetKwFunction{FMain}{Main}
    \SetKwProg{Fn}{Function}{:}{}
    \Fn{\FMain{}}{
         
         \\\text{Get $M$;}
          \\\text{Get $N$;}
          \\ \textbf{int} DP[M][N]={-1}
          \\\textbf{Print}
          \text{(TotalSequences($M$, $N$))}
        
         
}
 
\end{algorithm}\\
\\
\begin{algorithm}[H] 
    \caption{Algorithm 2: Dynamic Programming}
    \KwIn{two numbers \begin{math}
M\end{math} and \begin{math}
N\end{math}}
    \KwOut {Total number of possible sequences}
    \DontPrintSemicolon
    \SetKwFunction{FMain}{TotalSequences}
    \SetKwProg{Fn}{Function}{:}{}
    \Fn{\FMain{$int$ $M$, $int$ $N$}}{
    
    
    \\ \textbf{int} DP[M+1][N+1];\\
    \For{$i \gets 0$ \textbf{ to } $M$}
                 {
                  \For{$j \gets 0$ \textbf{ to } $N$}
                  {
                    \\ \textbf{if} i==0 || j==0;
                      \\ \textbf{then} DP[i][j] = 0;\\\\
                      
                      \\ \textbf{else if} i < j;
                      \\ \textbf{then} DP[i][j] = 0;\\
                      
                       \\ \textbf{else if} j == 1;
                      \\ \textbf{then} DP[i][j] = i;\\
                      
                       \\ \textbf{else}
                      \\ \text{DP[i][j] = DP[i-1][j] + DP[i/2][j-1];} 
                  }
                
                }
                 \\ \textbf{return}
                       \text{DP[M][N];} 
   }
    \SetKwFunction{FMain}{Main}
    \SetKwProg{Fn}{Function}{:}{}
    \Fn{\FMain{}}{
         
         \\\text{Get $M$;}
          \\\text{Get $N$;}
          \\ \textbf{int} DP[M][N]={-1}
          \\\textbf{Print}
          \text{(TotalSequences($M$, $N$))}
        
         
}
 
\end{algorithm}

\\




\section{TIME CALCULATION }\\
The total number of sequences is the sum of the number of sequences including m and the number of sequences where m is not included. Thus the original problem of finding number of sequences of length \begin{math}N\end{math} with max value \begin{math}M\end{math} can be subdivided into independent sub-problems of finding number of sequences of length n with max value \begin{math}M-1\end{math} and number of sequences of length \begin{math}N-1\end{math} with max value \begin{math}M/2\end{math}.
\begin{center}
\begin{tikzpicture}[trim left=0.3cm]
\begin{axis}[
    title={Complexity analysis for Dynamic programming approach},
    xlabel={Time taken to execute (ms)},
    ylabel={Value of \begin{math}N*M\end{math} (in thousands)},
     xmin=0, xmax=100,
    ymin=0, ymax=10,
    xtick={0,10,20,30,40,50,60,70,80,90,100},
    ytick={0,1,2,3,4,5,6,7,8,9,10}
]

\addplot[
    color=blue,
    mark=dot,
    ]
    coordinates {
    (0,0)(10,0.51)(20,1.03)(30,1.5)(40,2)(60,3)(100,5);
    };
    \legend{\begin{math}\mathcal{O}(N*M)\end{math} algorithm}
   \end{axis}
\end{tikzpicture}
\end{center} 

\section{TIME COMPLEXITY}\\
\begin{center}
 \begin{tabular}{||c c c||} 
 \hline
  BEST & AVERAGE & WORST CASE  \\  
 \hline\hline
 \begin{math} \mathcal{O}(M*N)\end{math} & \begin{math} \mathcal{O}(M*N)\end{math} & \begin{math} \mathcal{O}(M*N)\end{math} \\ 
 \hline
\end{tabular}
\end{center}

\section{CONCLUSION}
The above problem can be solved by using Dynamic Programming also. but the most efficient way is the divide and conquer.
We can conclude from our model that the running time
of our algorithm is O(M*N). We have tested our Algorithm
Theoretically and experimentally and both yielded similar results. 
the above problem can be solved by using Dynamic Programming also. but the most efficient way is the divide and conquer.

\section{REFERENCES}
\begin{enumerate}
    \item \url{https://www.geeksforgeeks.org/sequences-given-length-every-element-equal-twice-previous/}
    \item \url{https://tutorialspoint.dev/algorithm/dynamic-programming-algorithms/sequences-given-length-every-element-equal-twice-previous }
\end{enumerate}
\end{document}
